
\documentclass[a4paper,UKenglish]{dagman}
  %for A4 paper format use option "a4paper", for US-letter use option "letterpaper"
  %for british hyphenation rules use option "UKenglish", for american hyphenation rules use option "USenglish"
  %for section-numbered lemmas etc., use "numberwithinsect"

\usepackage{xspace}
\usepackage{microtype}%if unwanted, comment out or use option "draft"
% ============================================================
%:Markup macros for proof-reading
\usepackage{ifthen}
\usepackage[normalem]{ulem} % for \sout
\usepackage{xcolor}
\newcommand{\ra}{$\rightarrow$}
\newboolean{showedits}
\setboolean{showedits}{true} % toggle to show or hide edits
%\setboolean{showedits}{false} % toggle to show or hide edits
\ifthenelse{\boolean{showedits}}
{
	\newcommand{\meh}[1]{\textcolor{red}{\uwave{#1}}} % please rephrase
	\newcommand{\ins}[1]{\textcolor{blue}{\uline{#1}}} % please insert
	\newcommand{\del}[1]{\textcolor{red}{\sout{#1}}} % please delete
	\newcommand{\chg}[2]{\textcolor{red}{\sout{#1}}{\ra}\textcolor{blue}{\uline{#2}}} % please change
	\newcommand{\nbe}[3]{
		{\colorbox{#3}{\bfseries\sffamily\scriptsize\textcolor{white}{#1}}}
		{\textcolor{#3}{\sf\small$\blacktriangleright$\textit{#2}$\blacktriangleleft$}}}
}{
	\newcommand{\meh}[1]{#1} % please rephrase
	\newcommand{\ins}[1]{#1} % please insert
	\newcommand{\del}[1]{} % please delete
	\newcommand{\chg}[2]{#2}
	\newcommand{\nbe}[3]{}
}
%
\newcommand\rA[1]{\nbe{Reviewer A}{#1}{cyan}}
\newcommand\rB[1]{\nbe{Reviewer B}{#1}{olive}}
\newcommand\rC[1]{\nbe{Reviewer C}{#1}{magenta}}
\newcommand\ANS[1]{\nbe{Response}{#1}{teal}}
% ============================================================
%:Box comments/edits
\usepackage[most]{tcolorbox}
\ifthenelse{\boolean{showedits}}
{
  \newtcolorbox{inserted}{%
       title=Inserted text:,
       colframe=blue,colback=blue!5!white,
       breakable,
       leftrule=0mm, 
       bottomrule=0mm,
       rightrule=0mm,
       toprule=0mm,
       arc=0mm, outer arc=0mm,
       oversize
  }
  \newtcolorbox{deleted}{%
       title=Deleted text:,
       colframe=red,colback=red!5!white,
       breakable,
       leftrule=0mm, 
       bottomrule=0mm,
       rightrule=0mm,
       toprule=0mm,
       arc=0mm, outer arc=0mm,
       oversize
  }
  \newtcolorbox{refactored}{%
       % title=Heavily modifed/refactored text:,
       title=Rewritten text:,
       colframe=blue,colback=red!5!white,
       breakable,
       leftrule=0mm, 
       bottomrule=0mm,
       rightrule=0mm,
       toprule=0mm,
       arc=0mm, outer arc=0mm,
       oversize
  }
}{
  \newenvironment{inserted}{}{}
  %\newenvironment{deleted}{ \begin{comment} }{ \end{comment} }
  \let\deleted\comment
  \newenvironment{refactored}{}{} 
}
% ============================================================
%:Put edit comments in a really ugly standout display
%\usepackage{ifthen}
\usepackage{amssymb}
\newboolean{showcomments}
\setboolean{showcomments}{true}
%\setboolean{showcomments}{false}
\newcommand{\id}[1]{$-$Id: scgPaper.tex 32478 2010-04-29 09:11:32Z oscar $-$}
\newcommand{\yellowbox}[1]{\fcolorbox{gray}{yellow}{\bfseries\sffamily\scriptsize#1}}
\newcommand{\triangles}[1]{{\sf\small$\blacktriangleright$\textit{#1}$\blacktriangleleft$}}
\ifthenelse{\boolean{showcomments}}
%{\newcommand{\nb}[2]{{\yellowbox{#1}\triangles{#2}}}
{\newcommand{\nbc}[3]{
 {\colorbox{#3}{\bfseries\sffamily\scriptsize\textcolor{white}{#1}}}
 {\textcolor{#3}{\sf\small$\blacktriangleright$\textit{#2}$\blacktriangleleft$}}}
 \newcommand{\version}{\emph{\scriptsize\id}}}
{\newcommand{\nbc}[3]{}
 \newcommand{\version}{}}
\newcommand{\nb}[2]{\nbc{#1}{#2}{orange}}
\newcommand{\here}{\yellowbox{$\Rightarrow$ CONTINUE HERE $\Leftarrow$}}
\newcommand\rev[2]{\nb{TODO (rev #1)}{#2}} % reviewer comments
\newcommand\fix[1]{\nb{FIX}{#1}}
\newcommand\todo[1]{\nb{TO DO}{#1}}
\newcommand\on[1]{\nbc{Oscar}{#1}{olive}} % add more author macros here
\newcommand\jv[1]{\nbc{Jurgen}{#1}{red}}
\newcommand\cg[1]{\nbc{Carol}{#1}{blue}}
\newcommand\jh[1]{\nbc{James}{#1}{brown}}
\newcommand\ck[1]{\nbc{Claude}{#1}{cyan}}
   \definecolor{darkgreen}{rgb}{0,0.6,0}
\newcommand\katznote[1]{\nbc{Dan}{#1}{darkgreen}} % add more author macros here
   \definecolor{bluegreen}{rgb}{0,0.5,0.5}
\newcommand\kt[1]{\nbc{Kt}{#1}{bluegreen}}
%\newcommand\XXX[1]{\nbc{XXX}{#1}{darkgray}}
%\newcommand\XXX[1]{\nbc{XXX}{#1}{gray}}
%\newcommand\XXX[1]{\nbc{XXX}{#1}{olive}}
%\newcommand\XXX[1]{\nbc{XXX}{#1}{orange}}
%\newcommand\XXX[1]{\nbc{XXX}{#1}{purple}}
%\newcommand\XXX[1]{\nbc{XXX}{#1}{red}}
%\newcommand\XXX[1]{\nbc{XXX}{#1}{teal}}
%\newcommand\XXX[1]{\nbc{XXX}{#1}{violet}}
% ============================================================

% ============================================================
\renewcommand{\paragraph}[1]{\subsubsection*{#1}\xspace}
% ============================================================
\newcommand{\ie}{\emph{i.e.},\xspace}
\newcommand{\eg}{\emph{e.g.},\xspace}
\newcommand{\etal}{\emph{et al.}\xspace}
\newcommand{\etc}{\emph{etc.}\xspace}
% ==========================================================

\bibliographystyle{plain}%the recommended bibstyle

%Author macros: begin%%%%%%%%%%%%%%%%%%%%%%%%%%%%%%%%%%%%%%%%%%%%%%%%%%%%%
\subject{Manifesto for Dagstuhl Perspectives Workshop 16252}
\title{Manifesto on Engineering Academic Software}
% \titlerunning{A Manifesto Sample}%optional


\author[1]{Alice Allen}\affil[1]{University of Maryland -- College Park, US}
\author[2]{Cecilia Aragon}\affil[2]{University of Washington -- Seattle, US}
\author[3]{Christoph Becker}\affil[3]{University of Toronto, Canada}
\author[4]{Jeffrey Carver}\affil[4]{University of Alabama, US}
\author[5]{Andrei Chi\c{s}}\affil[5]{University of Bern, Switzerland}
\author[6]{Benoit Combemale}\affil[6]{IRISA -- Rennes, France}
\author[7]{Mike Croucher}\affil[7]{University of Sheffield, UK}
\author[8]{Kevin Crowston}\affil[8]{Syracuse University, US}
\author[9]{Daniel Garijo}\affil[9]{Technical University of Madrid, Spain}
\author[10]{Ashish Gehani}\affil[10]{SRI -- Menlo Park, US}
\author[11]{Carole Goble}\affil[11]{University of Manchester, UK}
\author[12]{Robert Haines}\affil[12]{University of Manchester, UK}
\author[13]{Robert Hirschfeld}\affil[13]{Hasso-Plattner-Institut -- Potsdam, Germany}
\author[14]{James Howison}\affil[14]{University of Texas -- Austin, US}
\author[15]{Katy Huff}\affil[15]{University of California -- Berkeley, US}
\author[16]{Caroline Jay}\affil[16]{University of Manchester, UK}
\author[17]{Dan Katz}\affil[17]{University of Illinois at Urbana Champaign, US}
\author[18]{Claude Kirchner}\affil[18]{INRIA -- Le Chesnay, France}
\author[19]{Katie Kuksenok}\affil[19]{University of Washington -- Seattle, US}
\author[20]{Ralf L\"{a}mmel}\affil[20]{Universit\"{a}t Koblenz-Landau, Germany}
\author[21]{Oscar Nierstrasz}\affil[21]{University of Bern, Switzerland}
\author[22]{Matt Turk}\affil[22]{University of Illinois at Urbana Champaign, US}
\author[23]{Rob van Nieuwpoort}\affil[23]{VU University Amsterdam, The Netherlands}
\author[24]{Matthew Vaughn}\affil[24]{University of Texas -- Austin, US}
\author[25]{Jurgen Vinju}\affil[25]{CWI -- Amsterdam, The Netherlands}

%
%\author[1]{Alice Allen}\affil[1]{University of Maryland -- College Park, US. \texttt{aallen@astro.umd.edu}}
%\author[2]{Cecilia Aragon}\affil[2]{University of Washington -- Seattle, US. \texttt{aragon@uw.edu}}
%\author[3]{Christoph Becker}\affil[3]{University of Toronto, Canada. \texttt{christoph.becker@utoronto.ca}}
%\author[4]{Jeffrey Carver}\affil[4]{University of Alabama, US. \texttt{carver@cs.ua.edu}}
%\author[5]{Andrei Chi\c{s}}\affil[5]{University of Bern, Switzerland. \texttt{andrei@inf.unibe.ch}}
%\author[6]{Benoit Combemale}\affil[6]{IRISA -- Rennes, France. \texttt{benoit.combemale@irisa.fr}}
%\author[7]{Mike Croucher}\affil[7]{University of Sheffield, UK. \texttt{m.croucher@sheffield.ac.uk}}
%\author[8]{Kevin Crowston}\affil[8]{Syracuse University, US. \texttt{crowston@syr.edu}}
%\author[9]{Daniel Garijo}\affil[9]{Technical University of Madrid, Spain. \texttt{dgarijo@isi.edu}}
%\author[10]{Ashish Gehani}\affil[10]{SRI -- Menlo Park, US. \texttt{ashish.gehani@sri.com}}
%\author[11]{Carole Goble}\affil[11]{University of Manchester, UK. \texttt{carole.goble@manchester.ac.uk}}
%\author[12]{Robert Haines}\affil[12]{University of Manchester, UK. \texttt{robert.haines@manchester.ac.uk}}
%\author[13]{Robert Hirschfeld}\affil[13]{Hasso-Plattner-Institut -- Potsdam, Germany. \texttt{robert.hirschfeld@hpi.de}}
%\author[14]{James Howison}\affil[14]{University of Texas -- Austin, US. \texttt{jhowison@ischool.utexas.edu}}
%\author[15]{Katy Huff}\affil[15]{University of California -- Berkeley, US. \texttt{katyhuff@gmail.com}}
%\author[16]{Caroline Jay}\affil[16]{University of Manchester, UK. \texttt{caroline.jay@manchester.ac.uk}}
%\author[17]{Dan Katz}\affil[17]{University of Illinois at Urbana Champaign, US. \texttt{dskatz@illinois.edu}}
%\author[18]{Claude Kirchner}\affil[18]{INRIA -- Le Chesnay, France. \texttt{claude.kirchner@inria.fr}}
%\author[19]{Katie Kuksenok}\affil[19]{University of Washington -- Seattle, US. \texttt{Katerena.Kuksenok@gmail.com}}
%\author[20]{Ralf L\"{a}mmel}\affil[20]{Universit\"{a}t Koblenz-Landau, Germany. \texttt{laemmel@uni-koblenz.de}}
%\author[21]{Oscar Nierstrasz}\affil[21]{University of Bern, Switzerland. \texttt{oscar@inf.unibe.ch}}
%\author[22]{Matt Turk}\affil[22]{University of Illinois at Urbana Champaign, US. \texttt{matthewturk@gmail.com}}
%\author[23]{Rob van Nieuwpoort}\affil[23]{VU University Amsterdam, The Netherlands. \texttt{rob@cs.vu.nl}}
%\author[24]{Matthew Vaughn}\affil[24]{University of Texas -- Austin, US. \texttt{vaughn@tacc.utexas.edu}}
%\author[25]{Jurgen Vinju}\affil[25]{CWI -- Amsterdam, The Netherlands. \texttt{Jurgen.Vinju@cwi.nl}}


%\author[1]{Long list of authors}
%%\author[2]{Joan R. Access}
%\affil[1]{Dummy University Computing Laboratory, Dummy Country
%  \texttt{open@dummyuni.org}}
%%\affil[2]{Department of Informatics, Dummy College Address, Country
%%  \texttt{access@dummycollege.org}}
\authorrunning{A. Allen et al.}%optional

\subjclass{Dummy classification: please check \url{http://www.acm.org/about/class/ccs98-html}. Cite, for example, as:  ``B.3.3 Performance Analysis and Design Aids, C.1.2 Multiple Data Stream Architectures (Multiprocessors)''}% mandatory: Please choose ACM 1998 classifications from http://www.acm.org/about/class/ccs98-html . E.g., cite as "F.1.1 Models of Computation". 
\keywords{Dummy keywords: Please provide 1--5 keywords}% mandatory: Please provide 1-5 keywords

\seminarnumber{10101}
\semdata{03.--07.~January, 2011 -- \href{http://www.dagstuhl.de/10101}{www.dagstuhl.de/10101}}
\additionaleditors{Anne Helper}%optional
%Author macros: end%%%%%%%%%%%%%%%%%%%%%%%%%%%%%%%%%%%%%%%%%%%%%%%%%%%%%

%Dagstuhl editorial office macros: begin%%%%%%%%%%%%%%%%%%%%%%%%%%%%%%%%%%%%%
\volumeinfo%(easychair interface)
  {John Q. Open and Joan R. Access}%editors
  {2}%number of editors
  {A Manifesto Sample}%event
  {1}%volume
  {1}%issue
  {1}%starting page number
\DOI{10.4230/DagMan.1.1.1}%(DagRep.<issue no>.<volume no>.<firstpage>)
%Dagstuhl editorial office macros: end%%%%%%%%%%%%%%%%%%%%%%%%%%%%%%%%%%%%%

\begin{document}

\maketitle

% ==================================================
\begin{abstract}

\todo{ABSTRACT < 1 page}

\end{abstract}

% ==================================================
%:=== Executive Summary ===
\section*{Executive Summary}
% \summaryauthor and \license is optional
%\summaryauthor[John Q. Open and Joan R. Access]{%
%John Q. Open\\
%Joan R. Access
%}
%\license

\todo{
1-2 pages TLDR.
}


\paragraph{Citation \& Reviewing}
\begin{itemize}
\item I will properly cite software used to produce my research results.
%\item When reviewing, I will require others to properly cite software used to produce research results.
%\item I will actively encourage funding agencies to include software experts in their review processes.
\end{itemize}

\paragraph{Recognition}
\begin{itemize}
\item I will recognize software contributions in hiring and promotion within my institution.
%\item I will recognize software contributions at conferences by proposing dedicated sessions and prizes.
\end{itemize}

\paragraph{Making intellectual content visible}
\begin{itemize}
\item I will publish the intellectual contributions of my research software.
%\item I will distinguish the intellectual contribution of my software from its service contribution.
%\item I will invite developers of software that enables research to be co-authors on papers about that research.
%\item I will publish how I organize and run my software projects.
\end{itemize}

\paragraph{Software Projects}
\begin{itemize}
\item I will develop software in the open from the start, whenever possible.
%\item I will acknowledge that reading and understanding source code is a legitimate part of the academic discussion.
%\item I will match proposed software engineering practices to the actual needs and resources of the project.
%\item I will help scientists improve the quality of their software without passing judgment.
%\item I will document my academic software, including usage instructions, and input and output examples.
\end{itemize}

\paragraph{Sustainability}
\begin{itemize}
\item I will contribute to sustaining software I use and rely on. 
%\item I will package, release and archive versions of my software.
%\item I will make explicit how to cite my software.
%\item I will consider and document the sustainability of my research software as thoroughly as its function.
\end{itemize}



% ==================================================
\tableofcontents

% ==================================================
%:=== Introduction ===
\section{Introduction}

\todo{EDIT THIS DOWN (from the proposal)}

We propose a perspective seminar on the future of academic software, specially the process of engineering academic software and the resulting software quality. The current reality is that software is used both as academic research results and as part of academic research methods. With the advent of open-source software, artifact evaluation committees of conferences, and journals which include source code and running systems as part of the published artifacts, we foresee that software will increasingly be part of the academic process. The quality of this software must be accounted for, both a priori and a posteriori.

In this context it is highly relevant that (a) we distribute software engineering knowledge and expertise across communities and (b) we identify  strengths, weaknesses, risks and opportunities of academic software engineering. Our goal is to produce a roadmap towards future professional software engineering for software-based research instruments and other software produced and used in an academic context.


\subsection*{Topics of the Seminar}

\paragraph{Academia is software driven.} As software is becoming a pervasive technology for automating and innovating every aspect of the human condition (work, play, love and war), it is also embedded firmly in the academic world. On the one hand, in computer science and software engineering research in particular we see experimental software and toolkits emerge continuously, either as part of the \emph{output} of research effort, or as part of the \emph{research method}. On the other hand, in general it may be that software is used even more actively in the other fields of research such as mathematics, biology, particle physics, astronomy, medicine, law. Again, we can distinguish the software part of the output (\ie development of innovative production techniques) from software which is part of research methods. 

\paragraph{The software $\times$ open-data flywheel effect} emphasises the urgency of a focus on quality in academic software engineering. There is an explosion of available open-data online which is accessed and analysed through the creation of new software ---generating more data to analyse. Setting aside the quality of the data itself (which is a major topic but out-of-scope for the current seminar), we focus on the impact of the quality of the software which processes it. We scrutinise any software which acquires, cleanses, stores, annotates, transforms, filters, generates (etc.) research data.

\paragraph{The perspective of this perspective seminar is that of the research team developing and/or using academic software.} As software is becoming integral to our processes, the tools we use and the output we produce, this perspective provides a starting point for a discussion both both timely and pressing:

\begin{itemize}
\item What is academic software? How is it different from other software? What are its most pressing dimensions of quality? What are the major success factors? What are common pitfalls?
\item Is the software that drives our research methods correct? Are the inputs and outputs sufficiently specified to be able to interpret the difference between incorrect and correct? How to verify (test or prove) our claims?
\item Is the software we use and produce in an academic context sustainable?\footnote{The opposite of sustainable software, often jokingly referred as ``PhD-ware'', is a serious threat to the rigor of the academic process.} This includes the matters of (1) software maintenance and (2) software evolution. Can we be assured to be able reproduce previous research methods in the future under arbitrary changes to the technological contexts (machines, operating systems, programming languages frameworks)? Are we able to incrementally adapt research software to emerging opportunities at the same time, without loss of reproducibility and without incurring prohibitive cost?
\item Is the software process we use fit for the quality we expect? How to optimize it in the unique academic context without losing quality? What tools and processes exist to help with this balance? What investments are necessary to find it?
\item How to secure academic software quality? How to monitor, steer, report on and review academic software quality? How to manage and secure trust between academic research teams considering software developed for output and or research methods?
\item How to balance domain knowledge and expertise with software engineering knowledge and expertise in an academic research team? How to manage heterogeneous research teams where both domains benefit from each other?   
\item What are motivators for investment and change for research teams, with respect to the above, considering the highly competitive and already complex environment they operate in? Is it clear what is required in terms of long term funding, education and infra-structure to make the goals of academic software feasible?
\end{itemize}

\subsection*{Goals of the Seminar}

\paragraph{Awareness, Synergy and Strategic Planning}

Although each research domain may be unique, there do exist common issues across the domains with respect to software. For example, there exists the common phenomenon of ``PhD-ware'', where software is developed, used, and thrown away within the lifecycle of PhD research projects rather than being reused and maintained for future projects. Another example is the widespread use of flexible scripting languages in research labs while state-of-the-art quality assessment and other supporting tools work best on statically typed and compiled languages.

Based on anecdotal evidence, we conjecture that many aspects of software engineering are the same or comparable between different research domains and that we can learn from each other. The seminar will provide an answer to the question whether this conjecture holds.

We believe that the software engineering research community is well-positioned to provide input for the other communities on the aspect of research software. At the same time, the software engineering community will learn from the other communities who have perhaps more experience in validating experimental research methods or calibrating research instruments. 

The goals of the seminar are to plan how to widen and deepen the impact of software engineering knowledge in research labs across the globe and to prioritize pressing open questions for the software engineering community with respect to research software.

\paragraph{Deliverables: Impact on the Research Community}

This seminar would be an element of an ongoing global effort to increase awareness and professional attitudes and skills in software engineering in research labs. We point to the Software Carpentry Foundation (SCF, \url{https://software-carpentry.org/}) and the Software Sustainability Institute (SSI, \url{http://www.software.ac.uk/}) of which key members contribute to the seminar.

The current seminar will be geared towards knowledge exchange and harvesting the communication between the participants in the following (short) draft documents:
\begin{itemize}
\item Ontology --- defines and explains briefly what we are talking about, how academic software relates to software engineering in general. It takes the Software Engineering Body of Knowledge (SWEBOK)\footnotemark \footnotetext{\url{https://www.computer.org/web/swebok}} and ISO/IEC 25010:\-2011\footnotemark~ on Software Quality as starting points; \footnotetext{\url{http://en.wikipedia.org/wiki/ISO/IEC_9126}}
\item SWOT analysis --- a summary of strengths, weaknesses, opportunities and risks in academic software engineering. Uses the jargon identified in the previous and documents our ongoing discussion;
\item Questionnaire --- using the previous results, we will design a questionnaire to establish the state-of-the-art in academic software engineering from the global community;
\item Manifesto --- we summarize the ontology and the SWOT analysis into a document of proposed steps towards improving the state-of-the-art in academic software engineering globally.
\end{itemize} 

These results will form a starting point for disseminating the lessons learned and best practices via different funding agencies at the national and international level, and also (informal) publications (like ERCIM News, ACM and IEEE magazines) and the respective magazines in the specific research domains. Both the SSI and the SCF will play a vital role in disseminating the results of the current seminar. We also intend to reach out to national interest groups such as the Netherlands SIG (Special Interest Group Software Engineering)


% ==================================================
%:=== PLEDGES ===
\section{Responsible Academic Software Development}

\todo{Pledges plus explanations}



\paragraph{Citation \& Reviewing}
\begin{itemize}
\item I will properly cite software used to produce my research results.
\item When reviewing, I will require others to properly cite software used to produce research results.
\item I will actively encourage funding agencies to include software experts in their review processes.
\end{itemize}

\paragraph{Recognition}
\begin{itemize}
\item I will recognize software contributions in hiring and promotion within my institution.
\item I will recognize software contributions at conferences by proposing dedicated sessions and prizes.
\end{itemize}

\paragraph{Making intellectual content visible}
\begin{itemize}
\item I will publish the intellectual contributions of my research software.
\item I will distinguish the intellectual contribution of my software from its service contribution.
\item I will invite developers of software that enables research to be co-authors on papers about that research.
\item I will publish how I organize and run my software projects.
\end{itemize}

\paragraph{Software Projects}
\begin{itemize}
\item I will develop software in the open from the start, whenever possible.
\item I will acknowledge that reading and understanding source code is a legitimate part of the academic discussion.
\item I will match proposed software engineering practices to the actual needs and resources of the project.
\item I will help scientists improve the quality of their software without passing judgment.
\item I will document my academic software, including usage instructions, and input and output examples.
\end{itemize}

\paragraph{Sustainability}
\begin{itemize}
\item I will contribute to sustaining software I use and rely on. 
\item I will package, release and archive versions of my software.
\item I will make explicit how to cite my software.
\item I will consider and document the sustainability of my research software as thoroughly as its function.
\end{itemize}


%:=== DISCUSSION OF PLEDGES ===
% --------------------------------------------------
\subsection{Citation \& Reviewing}

\paragraph{I will properly cite software used to produce my research results.}

\paragraph{When reviewing, I will require others to properly cite software used to produce research results.}

\paragraph{I will actively encourage funding agencies to include software experts in their review processes.}


% --------------------------------------------------
\subsection{Recognition}

\paragraph{I will recognize software contributions in hiring and promotion within my institution.}

\paragraph{I will recognize software contributions at conferences by proposing dedicated sessions and prizes.}


% --------------------------------------------------
\subsection{Making intellectual content visible}

\paragraph{I will publish the intellectual contributions of my research software.}

\paragraph{I will distinguish the intellectual contribution of my software from its service contribution.}

\paragraph{I will invite developers of software that enables research to be co-authors on papers about that research.}

\paragraph{I will publish how I organize and run my software projects.}


% --------------------------------------------------
\subsection{Software Projects}

\paragraph{I will develop software in the open from the start, whenever possible.}

\paragraph{I will acknowledge that reading and understanding source code is a legitimate part of the academic discussion.}

\paragraph{I will match proposed software engineering practices to the actual needs and resources of the project.}

\paragraph{I will help scientists improve the quality of their software without passing judgment.}

\paragraph{I will document my academic software, including usage instructions, and input and output examples.}

% --------------------------------------------------
\subsection{Sustainability}

\paragraph{I will contribute to sustaining software I use and rely on.}

\paragraph{I will package, release and archive versions of my software.}

\paragraph{I will make explicit how to cite my software.}

\paragraph{I will consider and document the sustainability of my research software as thoroughly as its function.}




% ==================================================
%:=== Future Research Directions ===
\section{Future Research Directions}

The research areas outlined below are concerned with gaining a deeper understanding of both the process and the output of research software engineering activities. Because both tools and practices change so rapidly, and because such a wide diversity of these exists across different scientific disciplines and contexts, the overarching challenge is to understand how findings about specific projects or groups relate to broader trends over time. The research agenda includes developing ways to measure meaningful aspects of both practice and output of scientific programming and questions regarding design and development of tools to support aspects of relevant activities.

% --------------------------------------------------
\subsection{Quantifying the state of scientific software availability.}

Existing work has included literature surveys of particular publication venues [e.g., TODO citations] which aim to provide some measurement of how much software is available. However, there is limited consensus on what constitutes code that is expected to be released, and the level of scrutiny with which to evaluate a particular publication. These limitations make comparisons across venues - let alone fields or disciplines - difficult. Such comparisons constitute a component of understanding the change which we try to bring about with improved tools and advocacy programs. 

Specific questions:
\begin{itemize}
\item Map out which stakeholders (institutional, funding, etc) are invested in this information, and how to support the systematic gathering of relevant data. Map out existing work that has done these kinds of surveys and collate the various means to measure code availability and create a consistent recommended primer for conducting this survey, in order to enable a broader scope of study and comparability.
\item In published scientific literature, how are rates of available, runnable, and hidden software changing over time?
\item How can code duplication be identified? What are reasons for duplicated code, and what proportion of these are the result of breakdown in code discovery (rather than a reason arising from other needs)?
\end{itemize}

% --------------------------------------------------
\subsection{Sustainability of software experimentation}

Reproducibility of scientific experiments with major computational components require set-up environments that will, in principle, allow experiments to be redone in 2, 10, 100 and perhaps even 1000 years. We envision a future in which programs should be rerunnable and feel as persistent as email seems today. A journal like ipol allows for example to publish a paper as well as the runnable program, at least on the current machine available in 2016. Jupyter Notebooks allow the bundling of all necessary components and dependencies. To extend the persistence of software execution across environments and tools opens new research tracks in the rapidly changing hardware and software environments that we currently living with. This requires to explore in particular the following specific questions. 

\begin{itemize}
\item What we would like to have for a research environment for reproducibility, including  software and data, in 10 years from now? 
\item How to design and implement virtual machine for forever-sustainable executable environments? 
\item How to deal with legacy of softwares on the long term, in particular

\begin{itemize}
\item De-optimization, porting, refactoring of academic software
\item Translation of legacy data formats?
\end{itemize}

\item Related work

\begin{itemize}
\item Executable papers, Peter Sloot (Elsevier?s executable paper competition)
\item SHARE [2.0] Pieter van Gorp
\end{itemize}

\end{itemize}


% --------------------------------------------------
\subsection{Facilitating software discovery within and across disciplines}

A major challenge of RSE is the diversity of programmer skill, project scale, level of concern and scale of action. Understanding the metadata ecosystem is another major research focus, but what other approaches, aside from (or complementary to) standards can support scaling discoverability? Other questions on this agenda are concerned with measuring the amount and cost of hidden software, but we already recognize it as a pain point that may be addressed with design, tooling, and workshops.


\begin{itemize}
\item HCI/design research: Understand user intention to support discoverability, explore opportunities for machine learning application
\item Instrumentation of existing infrastructure (github), or use of annotation infrastructure such as hypothes.is, json-ld
\end{itemize}

% --------------------------------------------------
\subsection{Software engineering tools improving productivity by tailoring to intent and skill}

Many common tools that embody SE best practices are not written for scientists. Tools like Jupyter Notebook provide a more natural environment. In much SE research, the assumption is that the programmer is building and maintaining an artefact, but many scientists write code that should work and be correct, but not necessary constitute the same scale or persistence of artefact. There are ongoing SE research projects in this direction. SE is a scholarly community well-positioned to understand and design for these forms of programming.

\begin{itemize}
\item How can tool design bring what SE already knows (e.g., about testing, debugging, reproducibility) to bear into the scientific context?
\item How are different tools augmenting programmer cognition?
\item Develop typologies of software and corresponding design requirements. For example:
\item Support of ?what-if? scenarios
\item Multi-scale interaction - capability to ?zoom? in and out of levels of abstraction
\item Avoid independent co-evolution
\end{itemize}

% --------------------------------------------------
\subsection{Econometrics of infrastructure and hidden software}

Claims of cost of hidden software or infrastructure maintenance remain (often) anecdotal or, at best, qualitative self-report. It is difficult to even measure the amount of hidden code (see Q1) let alone quantify its impact. However, this is needed for evaluation of interventions.


\begin{itemize}
\item What is the cost of code that is undiscoverable, unavailable, or unrunnable?
\item How much does it cost to support / not support software that others use? 
\item Map out existing models to share the costs, to inform recommendations
\end{itemize}


% --------------------------------------------------
\subsection{Baseline practices and product quality as well as their expected changes}


\begin{itemize}

\item Find out what ?normally? exists  and changes to be able to observe the effect of intended and unintended interventions
\item ?Avoid the base rate fallacy?
\item Ties in with question 1 (which first needs answering before we can go into this properly)
\item This requires frames of reference, such as:

    \begin{itemize}
    \item suitable quality models for research software
    \item level of training of developers
    \end{itemize}

\item Comparison with broader scopes such as open-source software
\item Ethnographic and qualitative studies
\end{itemize}

% --------------------------------------------------
\subsection{Scientist Centric and Model Centric Scientific Software Engineering}


\begin{itemize}
\item Although it is already like that informally, we need tooling and methodology to enable scientists to think about their models first class
\item Enabling co-research of the (visual) language needed to describe new theory, and enabling immediate simulation and verification of the new language.
\item A generalized/moldable matlab, moldable maple;
\end{itemize}

% --------------------------------------------------
\subsection{What is the invisible work of engineering academic software?}


\begin{itemize}
\item The human side
\item Why is it persistently not acknowledged
\end{itemize}

% --------------------------------------------------
\subsection{Metaphor, language, and contextuality}

% --------------------------------------------------
\subsection{Software as communication}

 ????


% --------------------------------------------------
\subsection{What?s different about research software outside the sciences - e.g. in humanities?}

(a fuzzy question at this point)

% --------------------------------------------------
\subsection{Analysis of scientific software ecosystem metadata}

With better metadata about projects we can pursue systems that provide insight into the scientific software ecosystem. Example efforts currently include Depsy (Piowawr and Priem) and the Scientific Software Network Map (Bogart, Howison, and Herbsleb). These take data on software, authors, mentions in publications, and software projects and compose them into maps that provide information on direct and indirect impact, as well as data often hidden from developers (such as which projects are used with others by end users).

Specific questions in the area of ecosystem analysis and metadata are:

\begin{itemize}
\item What kind of metadata is needed? What formats are appropriate?
\item Metadata: what motivates projects to generate and maintain metadata?
\item How well can software be observed in the literature?
\item What are the moments in which participants are well motivated (e.g., people are motivated just after publishing a new paper to add it to their request for citation).
\item Can we motivate software developers to improve metadata by pushing 
\item Metrics of software: how can quantify and measure code meaningfully
\end{itemize}

% --------------------------------------------------
\subsection{Techniques/metrics for evaluating the likely success and impact of proposed / nascent academic software projects}

% --------------------------------------------------
\subsection{Re-tooling the bibliographic software toolchain for software citation}

For the vision of software citation to be realized, authors need to be able to store a metadata record for software with the needed fields, have the toolchain pass these fields through, and have style files produce output that includes them. To achieve this, interfaces for metadata storage will need to be upgraded to meet these needs and improved style files will need to be pushed into the hands of users.

There are three broadly used bibliographic toolchains: bibtex, citation style language (Zotero | Papers | Mendeley ? citeproc ? Word | pandoc | Pages), and Endnote. Note, though, that there are many other systems:
\url{https://en.wikipedia.org/wiki/Comparison_of_reference_management_software}

In the bibtex world, this means standardizing a type definition (e.g., @software, although there may be existing partial standards) and have bibtex interfaces (such as bibdesk) handle it properly. The modern implementations of bibtex pass through unknown types, so that is fine. Then the bst files need to know what to do when receiving an @software record.

In the Citation Style Language (csl) world, used by Zotero, Mendeley, and Papers, we need a csl type that is well support by the user interfaces, the citeproc implementations to pass it through to the csl files (e.g., https://github.com/citation-style-language/styles) and the csl files to know what to do with it.

In the Endnote world Endnote and it?s proprietary style system makes some things difficult to change from the outside. Endnote has to have an appropriate type, and the style files have to process it.  Some information on this is available here: http://endnote.com/sites/en/files/m/pdf/en-x7-win-editing-reference-types-styles.pdf

We can define a set of example software citations and produce metadata records for them. We can then use a continuous integration tool to run each of those through each style in each bibliographic tool chain. We can then compare the outputs to the software citation principles and say whether the style file produces valid software citations. By automating that comparison we can define a test.

In the csl world style file distribution and editing is centralized; by changing the style files in the github repository they will eventually be distributed to users via their software client.

In the bibtex world, however, bst are not centralized but distributed in a heterogenous manner (partly through latex distributions, partly through publisher websites). Often the closest thing to a canonical distribution point is the publishers sites, so we would have to convince each publisher to take an edited bst file.  One approach to drive this change could be to regularly tweet the failing test to the publisher and provide a link to a file that does not fail the tests.

One additional necessary element would be tool support to make it easy to get appropriate metadata records into bibliographic software. For example, one could write web importers for Zotero that created an appropriate metadata record from the software landing page, with a generic fallback that works on repository home pages. These importers could read data from CITATION files, micro-format embedded metadata (e.g., schema.org SoftwareApplication) or data composed from repository APIs.




% ==================================================
%:=== APPENDIX ===
\begin{appendix}
\section{Related Material}

2. Pointers to, and commentary on, existing manifestos
The open science and research software communities have been very active in creating manifestos in the style of calls to action. In this section, we will pull these together and provide some commentary on them, based on our discussions this week.

Some such manifestos have calls for improved software and bibliography metadata for persistent citation of software. These include both the FAIR guiding principles and the Force11 Software Citation Principles and the Science Code Manifesto \cite{wilkinson_fair_2016,arfon_m._smith_software_2016,nick_barnes_science_2013}.

Other topics addressed by such manifestos include emphasis on access to source code, which was also a topic of interest in this workshop. Manifestos that address this topic include the Science Code Manifesto, the Karlskrona Manifesto for Sustainability Design, . 

Emphasis on grass-roots support and institutional change toward visibility of the research contributions of software and software engineers are also addressed by many of these manifestors. A domain-specific manifesto, Astronomical Software Wants To Be Free: A Manifesto, is a key example of a call to arms for that community to value the software underlying their science \cite{weiner_astronomical_2009}. Broader calls for this community action are included in the  UK RSE objectives and the Science Code Manifesto.

Many are more focused on the practice of reproducible software development for research. These inclue the Open Science Peer Review Oath, the Reproducibility PI Manifesto, the GeoScience Paper of the Future Initiative, and the FAIR principles. 


\paragraph{Science Code Manifesto}
\url{http://sciencecodemanifesto.org/}\cite{nick_barnes_science_2013}
Focuses on ``source code written to specifically process data for a published paper.''
Therefore, may only cover a portion of the types of software of interest (e.g. from the wording it appears that software that produces scientific data may not be included here).
It does cover Code, Copyright, Citation, Credit, and Curation.
The principles do not seem to be in conflict with our discussions this week

\paragraph{Force 11 Software Citation principles}
draft at \url{https://www.force11.org/software-citation-principles}
\cite{arfon_m._smith_software_2016}
The citation principles emphasize persistence and clarity.

\paragraph{Open Access Pledge}
\url{http://www.openaccesspledge.com/}
\cite{alex_holcombe_open_2011}
Focuses on promising to publish software and papers in open access venues.
Also notes an emphasis on contributing more.

\paragraph{Open Science Peer Review Oath}
\url{http://f1000research.com/articles/3-271/v2}
Focuses on leveraging one's power as a reviewer to demand open software access, reproducible practices, and transparent review ideals.

\paragraph{Karlskrona Manifesto for Sustainability Design}
\url{http://sustainabilitydesign.org/}
\cite{becker_karlskrona_2014}
Sustainable design in this context includes a broad definition of sustainability. In the technical sense, this is along the lines of WSSSPE, but it goes well beyond, including the environmental and social interpretations of sustainability.

\paragraph{Astronomical Software Wants To Be Free: A Manifesto}
\url{http://arxiv.org/abs/0903.3971}
\cite{weiner_astronomical_2009}
This one emphasized various principles that would improve the visibility and esteem of software within astronomy. It emphasizes structural incentives and adoption of values which recognize this. 

\paragraph{UK RSE}
About RSE: \url{http://www.rse.ac.uk/who.html}
\cite{rse_conference_2016_what_2016}
Objectives of UKRSE: \url{http://www.rse.ac.uk/objectives.html}
This emphasizes the raising of awareness of the role of Research Software Engineers through communication and support of institutional incentives.

\paragraph{Reproducibility manifesto}
\url{http://lorenabarba.com/gallery/reproducibility-pi-manifesto/}
\cite{barba_reproducibility_2012}
This includes terms to make software reusable by others. Focused on reproducibility, leaving sustainability of software out of the question.

\paragraph{The GeoScience paper of the future initiative}
\url{http://www.ontosoft.org/gpf/what-is-a-gpf}
\cite{onto_soft_what_2016}
Has a set of requirements for software to be included in a paper: ``with documentation, a license for reuse, and a unique and citable persistent identifier''. Differences with the scope of our seminar: it involves also data and its provenance, focusing more on the paper itself rather than the software.

\paragraph{FAIR principles}
\url{http://www.nature.com/articles/sdata201618}
\cite{wilkinson_fair_2016}
This includes focus on research data. The goal is to make them findable, accessible, interoperable and reusable. Can we apply these principles to software too?



% ==========================================================
%:=== JUNK ===
\section*{Brainstorming detritus}

NB: keep until final version, then delete

Other outputs that might be the basis for an I will:

Empirical survey of practices
Shoot the dogma
How to design communities
Teaching how to design
Teaching activities towards appreciation of software

Q: should we make calls to policy makers, as well? Calls for things that are not possible through our actions, that require change by others? 

Meta or follow-up manifesto?

research on software code manifesto?
\url{http://sciencecodemanifesto.org/}



\end{appendix}

% ==================================================
\begin{participants}
\on{Since everyone is an author, do we need this list?}\\
\participant Alice Allen\\ University of Maryland -- College Park
\participant Cecilia Aragon\\ University of Washington -- Seattle
\participant Christoph Becker\\ University of Toronto
\participant Jeffrey Carver\\ University of Alabama
\participant Andrei Chi\c{s}\\ University of Bern
\participant Benoit Combemale\\ IRISA -- Rennes
\participant Mike Croucher\\ University of Sheffield
\participant Kevin Crowston\\ Syracuse University
\participant Daniel Garijo\\ Technical University of Madrid
\participant Ashish Gehani\\ SRI -- Menlo Park
\participant Carole Goble\\ University of Manchester
\participant Robert Haines\\ University of Manchester
\participant Robert Hirschfeld\\ Hasso-Plattner-Institut -- Potsdam
\participant James Howison\\ University of Texas -- Austin
\participant Katy Huff\\ University of California -- Berkeley
\participant Caroline Jay\\ University of Manchester
\participant Dan Katz\\ University of Illinois at Urbana Champaign
\participant Claude Kirchner\\ INRIA -- Le Chesnay
\participant Katie Kuksenok\\ University of Washington -- Seattle
\participant Ralf L\"ammel\\ Universit\"at Koblenz-Landau
\participant Oscar Nierstrasz\\ University of Bern
\participant Matt Turk\\ University of Illinois at Urbana Champaign
\participant Rob van Nieuwpoort\\ VU University Amsterdam
\participant Matthew Vaughn\\ University of Texas -- Austin
\participant Jurgen Vinju\\ CWI -- Amsterdam
\end{participants}


% ==========================================================
%\section*{Acknowledgements}
%
%We thank John Doe and Jane Doe for their valuable contributions.


% ==========================================================
%\bibliography{dagman-sample} % use bibtex or thebibliography-environment (as below)
\bibliography{eas}

\end{document}
