% !TEX root = dagstuhl-eas-manifesto.tex
% ==========================================================
\section*{Incomplete research directions}

% --------------------------------------------------
\subsection{What is the invisible work of engineering academic software?}


\begin{itemize}
\item The human side
\item Why is it persistently not acknowledged
\end{itemize}

% --------------------------------------------------
\subsection{Metaphor, language, and contextuality}

% --------------------------------------------------
\subsection{Software as communication}

 ????


% --------------------------------------------------
\subsection{What's different about research software outside the sciences --- e.g. in humanities?}

(a fuzzy question at this point)

% --------------------------------------------------
\subsection{Techniques/metrics for evaluating the likely success and impact of proposed / nascent academic software projects}


% --------------------------------------------------
\subsection{Baseline practices and product quality as well as their expected changes}


\begin{itemize}

\item Find out what ``normally'' exists  and changes to be able to observe the effect of intended and unintended interventions
\item ``Avoid the base rate fallacy''
\item Ties in with question 1 (which first needs answering before we can go into this properly)
\item This requires frames of reference, such as:

    \begin{itemize}
    \item suitable quality models for research software
    \item level of training of developers
    \end{itemize}

\item Comparison with broader scopes such as open-source software
\item Ethnographic and qualitative studies
\end{itemize}
% ==========================================================
\section*{Brainstorming detritus}

NB: keep until final version, then delete

Other outputs that might be the basis for an I will:

Empirical survey of practices
Shoot the dogma
How to design communities
Teaching how to design
Teaching activities towards appreciation of software

Q: should we make calls to policy makers, as well? Calls for things that are not possible through our actions, that require change by others? 

Meta or follow-up manifesto?

research on software code manifesto?
\url{http://sciencecodemanifesto.org/}
